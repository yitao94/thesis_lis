\documentclass{listhesis}
% --- Listhesis builds on KOMA script report (scrreprt).
%     Class arguments are passed to that class.

% --- Add additional packages here using \usepackage{package-name}.

% --- Provide your thesis details here.
\setup{%
  % de,                   % uncomment if your thesis is in German
  author=Firstname Familyname, % your name
  title={Design and Implementation of an Intuitive and Flexible Hardware
    Accelerator for an Arbitrary Application},
  % date={July 14, 2017}, % submission date (today is used if unset)
  % type=bachelor,        % thesis type [master, bachelor, research, internship, diplom]
  advisor=Your Advisor, % your advisor (typically some PhD. student)
  supervisor=Prof. Dr. sc. techn. Andreas Herkersdorf, % your supervisor
  % % uncomment the next lines if your thesis was carried out in industry
  % company=External Company,
  % externalAdvisor=Your Companyadvisor
}

\begin{document}

\maketitle
\cleardoublepage

% --- Thesis abstract.
%     For German thesis also provide an English version via the optional
%     argument: \abstract[English]{German}
\abstract{An abstract is defined as an abbreviated accurate
  representation of the contents of a document. -- American National
  Standards Institute (ANSI)}
\makeabstract
\clearpage

% --- Content tables.
\tableofcontents
\clearpage
\listoffigures
\clearpage
\listoftables
\clearpage

% --- Your thesis starts here.
%     Use \chapter{}, \section{}, \subsection{}, \subsubsection{},
%     and \paragraph{} to structure your thesis.

\chapter{Introduction}
Reading this section should give the reader a understanding why he or
she should read your thesis and what he or she can expect from it. The
section should include the following parts.

\paragraph{Motivation}
What problem lead to the thesis? What's the context? What are the
challenges?

\paragraph{State of the art}
What have others done to solve a similar problem? (keep this short, more
to come in the following chapter)

\paragraph{Goal of this thesis}
Be as specific and measurable as possible.

\paragraph{Steps required to reach this goal}
Which steps are included in your thesis and are part of your task?
Which are not?

\paragraph{Your Method}
How will your reach those goals? Usually, this is equal to an outline of
the following chapters.

\chapter{Related Work}
Include only work that is addressing the same or a very similar goal as
yours.  First, explain what the other authors did in their work in your
own words. Then discuss why you solved the same problem differently.

\chapter{Approach}
The main part of your thesis. Here you describe your approach to solve
the problem you stated in the Introduction.  Be as general as possible.

\chapter{(Prototype) Implementation}
In many cases, you implemented your approach in some way. In this
chapter, you explain your implementation.  Usually you can't explain
every detail. The source code that you submit already contains the tiny
details of your implementation: you can skip them in the
description. Instead, focus on what was hard why you implemented it in a
particular way, not how you implemented it

\chapter{Evaluation}
Using the previously presented implementation, you can now evaluate your approach.\\
Get as many quantitative measurements as possible.  E.g. compare
different implementations in terms of hardware usage, latency...\\
This chapter is usually filled with graphs and numbers.  In the end, the
numbers should show how your approach is better (in some regard) than
existing approaches.

\chapter{Conclusion and Outlook}
\paragraph{Conclude}
Summarize the work you did. Show how well you reached the goal you
specified in the introduction.

\paragraph{Outlook}
Which new questions arise based on your results? If you would continue
to work on this topic, what would your next work item be?

% --- Bibliography
\cleardoublepage
\bibliographystyle{plain}
\bibliography{thesis}

% --- Mandatory confirmation.
\confirmation

\end{document}

%%% Local Variables:
%%% mode: latex
%%% TeX-master: t
%%% End:
